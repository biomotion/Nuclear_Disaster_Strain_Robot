\begin{abstract}

無人載具為各國之國防科技發展重點,本計畫重點在無人載具之軟硬體系統開發整合,在過去研究中,將演算法部署到各種實體無人載具為重要工作,但往往因系統版本、硬體規格等執行不易,特別是在各種戶外場域。本研究使用機器人作業系統(Robot Operating System, ROS),為近年推展之ROS-Military基礎,鑑於近年美國國防高等研究計劃署 (Defense Advanced Research Projects Agency, DARPA)使用Gazebo進行機器人競賽,本計畫首先建立機器人模型於Gazebo虛擬平台,進而設計立體視覺感測器,分析點雲及開發演算法於地形模擬建構,以開發適合崎嶇地形之運動規劃。最後,在微小化實體平台進行測試,未來將進一步部署於大型無人載具進行實際場域驗證,如戶外斜坡/草地/道路之沿路自主導航測試,本計畫也將所設計的演算法與感測裝置放置於水面無人載具,進行機器人避障自主導航測試,並將參與2018年在夏威夷舉行 的RobotX機器人競賽。 

關鍵詞:無人載具、自主導航、RobotX競賽。

Autonomous robotic software and hardware systems have been developed in military research in the field for decades. Recently the rise of robot operating system (ROS) facilitates the development (ROS-Military) across academia, industry, and government organizations. Nevertheless, the challenges making software packages of autonomous navigation available to unmanned vehicles remain, especially in the outdoor fields. This work aims to develop essential functionalities of autonomous navigation, from simulation (Gazebo) to real-world rough terrains and water/ocean environments. We develop a method for mapping uneven surfaces and obstacles with triangular meshes, given the fusion of point clouds over time. We compared several motion planning approaches in efficiency and desired tasks. We have deployed the software on a miniaturized mobile platform in a controlled environment, and will further test the system using full-sized unmanned ground and surface vehicles. Some of the work will be used for the Team NCTU participations of the RobotX competition in Hawaii in 2018.

Keywords: Unmanned System, Autonomous Navigation, RobotX Challenge

\end{abstract}
