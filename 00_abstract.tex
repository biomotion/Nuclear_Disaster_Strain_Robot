\begin{abstract}

目前,全球機器人市場持續增長,全球機器人產業市場規模逾270億美元,但開發目標卻是讓機器人執行人類原本就能完成的事,如說話、跳舞、拉小提琴,甚至主持婚禮,而人類做不到的事。
根據世界核能協會2012年8月的數據,全世界31個國家有435座工作反應爐。在2011福島核泄漏時,美國「國防部高級研究計劃局」(Defense Advanced Research Projects Agency,簡稱DARPA )的研究人員也參與了救援計劃,他們深感機器人在救災方面的局限。
本團隊此次主題將針對,核災現場以及人類無法到達之地區進行搜救、探勘,運用AI影像辨識技術及定位系統,結合自走車相關技術,進入高輻射區域建置現場地圖並標示出相關物體(生還者、燃料棒)之位置。\\

潛在使用者為世界上任一擁有核能發電廠之國家。\\

關鍵詞:無人載具、影像辨識、UWB定位、人本專題。\\
\end{abstract}


