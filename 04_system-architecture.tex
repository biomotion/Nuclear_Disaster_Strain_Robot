\section{系統設計}

\subsection{虛擬環境Gazebo}

模擬環境已經廣泛的使用於機器人之實驗測試,相較於使用實體機器人,模擬不只成本低能夠避免硬體受損,還能夠架設特殊場景,另外,可以測試多種不同假設情況,像是不同空氣阻力、水流方向、及機器人控制。Gazebo模擬平台可以提供複雜的環境建置並且準確有效的模擬訓練機器人,他擁有一個強大的物理引擎、先進的三維渲染、支援多種感測器、與ROS有極高的相容性,DARPA舉辦的競賽從早期的DARPA Urban Challenge ~\cite{ozguner2008simulation}、DARPA Robotics Challenge (DRC)、DARPA SubT Challenge,Gazebo均扮演重要角色,其他競賽如:NASA Sapce Robotics Challenge (SRC)、Robocup ~\cite{laue2005simrobot}、及Virtual Maritime RobotX Challenge (VMRC)~\cite{vmrc}也使用Gazebo環境。VMRC是與RobotX Maritime Challenge同樣團隊所開發的模擬競賽,透過VMRC平台所積極開發的模擬環境,不只提供開源程式碼定義基礎環境參數及USV建模及設定。本研究使用過去競賽(SRC, VMRC)所開放之環境為起點,加入相對應之機器人模型、感測器模型等,並對環境內障礙物等做進一步設計,供演算法開發使用。

\begin{figure}[h] % h means put this image here
\includegraphics[width=0.8\columnwidth]{images/gazebo.png}
\centering
\caption{Gazebo虛擬環境開發}
 \label{figure:gazebo}
\end{figure}

\subsection{分析點雲及開發演算法於地形模擬建構}

本論文的主要工作為融合時間區間的點雲資料來建置緊密地圖(Dense Mapping),可克服單一時間點點雲可能破碎、資料不足的缺點,相關方法已在文獻回顧介紹,本研究主要以Spatial Mapping~\cite{zed-spatial-mapping}方式進行測試,測試環境為交大竹湖旁之步道,周圍有樹叢斜坡、草地等非平坦地形,結果以三角網格(Triangular Mesh)呈現於圖~\ref{figure:dense-mapping}。

\begin{figure}[h] % h means put this image here
\includegraphics[width=\columnwidth]{images/dense-mapping.png}
\centering
\caption{地形建模,綠色部分為使用三角網格(Triangular Mesh)建立的非規則地形。}
 \label{figure:dense-mapping}
\end{figure}

\subsection{機器人定位與運動規劃}

本研究主要利用光達(LiDAR)所得到的點雲來做為船隻分析障礙物的感測資料,在取得障礙物資訊的部分,首先,我們利用雜訊濾波器濾除掉雜訊,接著我們使用隨機抽樣一致法(RANSAC)濾除掉海平面所反射的點雲資料,只留下水上的障礙物點雲資料。
在定位的部分,我們利用慣性測量單元(IMU)、全球定位系統(GPS)兩個感測器的資料,經由擴展卡曼濾波器(Extended Kalman Filter),取得優化後的船隻3D姿態描述資訊: 位置、轉向、速度,並將這些資料藉由ROS的訊息及命令服務,讓我們可以將這些資訊在RViz上視覺化出來,方便我們即時的了解船隻的動態。接著我們利用~\cite{Zhuang2005real}這篇論文的最小角度路徑規劃法實現路徑規劃演算法,他所利用的原理就是先將起點及終點連成一條線,接著檢查此線上是否有障礙物,若有的話,則計算出向障礙物左側或右側走的角度較小,而走那一側,藉此躲避障礙物。

我們比較快速隨機擴展樹(Rapidly-exploring Random Trees, RRT)~\cite{ABry2011RRT}及RRT-connect~\cite{JJKuffner2000RRTConnect} 兩種路徑規劃方法,在路徑規劃時間計算部分,我們發現利用RRT所需要計算的時間遠遠高於最小角度規劃法,因為RRT需要不斷地產生隨機節點作為路徑判斷,而最小角度法只需要在有障礙物的地方產生最小角度的路徑節點就好。另外在路徑流暢性的部分,RRT會在快靠近障礙物時才會開始轉彎,雖然可以大幅縮短路徑,但這對於在海上有極大慣性的無人載具來說是一大風險,而最小角度法則是可以在一開始就朝向無障礙物的路徑前進,可以更安全的避免碰撞障礙物。

最後在導航的部分,我們先利用橫麥卡托投影(UTM projection),將GPS絕對經緯度座標,轉換成適合無人載具的X、Y平面座標,藉此就可以指定一處經緯度座標為目標,讓無人載具利用前面的方法,在避障的情況下,順利抵達相對應的X、Y目標點。